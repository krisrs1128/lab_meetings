\documentclass{beamer}
\usetheme{Warsaw}
\usepackage{natbib}
\usepackage{graphicx}
% ------------------------------------------------------------------------
% Packages
% ------------------------------------------------------------------------
\usepackage{amsmath}

% ------------------------------------------------------------------------
% Macros
% ------------------------------------------------------------------------
%~~~~~~~~~~~~~~~
% List shorthand
%~~~~~~~~~~~~~~~
\newcommand{\BIT}{\begin{itemize}}
\newcommand{\EIT}{\end{itemize}}
\newcommand{\BNUM}{\begin{enumerate}}
\newcommand{\ENUM}{\end{enumerate}}
%~~~~~~~~~~~~~~~
% Text with quads around it
%~~~~~~~~~~~~~~~
\newcommand{\qtext}[1]{\quad\text{#1}\quad}
%~~~~~~~~~~~~~~~
% Shorthand for math formatting
%~~~~~~~~~~~~~~~
\newcommand\mbb[1]{\mathbb{#1}}
\newcommand\mbf[1]{\mathbf{#1}}
\def\mc#1{\mathcal{#1}}
\def\mrm#1{\mathrm{#1}}
%~~~~~~~~~~~~~~~
% Common sets
%~~~~~~~~~~~~~~~
\def\reals{\mathbb{R}} % Real number symbol
\def\integers{\mathbb{Z}} % Integer symbol
\def\rationals{\mathbb{Q}} % Rational numbers
\def\naturals{\mathbb{N}} % Natural numbers
\def\complex{\mathbb{C}} % Complex numbers
\def\simplex{\mathcal{S}} % Simplex
%~~~~~~~~~~~~~~~
% Common functions
%~~~~~~~~~~~~~~~
\renewcommand{\exp}[1]{\operatorname{exp}\left(#1\right)} % Exponential
\def\indic#1{\mbb{I}\left({#1}\right)} % Indicator function
\providecommand{\argmax}{\mathop\mathrm{arg max}} % Defining math symbols
\providecommand{\argmin}{\mathop\mathrm{arg min}}
\providecommand{\arccos}{\mathop\mathrm{arccos}}
\providecommand{\asinh}{\mathop\mathrm{asinh}}
\providecommand{\dom}{\mathop\mathrm{dom}} % Domain
\providecommand{\range}{\mathop\mathrm{range}} % Range
\providecommand{\diag}{\mathop\mathrm{diag}}
\providecommand{\tr}{\mathop\mathrm{tr}}
\providecommand{\abs}{\mathop\mathrm{abs}}
\providecommand{\card}{\mathop\mathrm{card}}
\providecommand{\sign}{\mathop\mathrm{sign}}
\def\rank#1{\mathrm{rank}({#1})}
\def\supp#1{\mathrm{supp}({#1})}
%~~~~~~~~~~~~~~~
% Common probability symbols
%~~~~~~~~~~~~~~~
\def\E{\mathbb{E}} % Expectation symbol
\def\Earg#1{\E\left[{#1}\right]}
\def\Esubarg#1#2{\E_{#1}\left[{#2}\right]}
\def\P{\mathbb{P}} % Probability symbol
\def\Parg#1{\P\left({#1}\right)}
\def\Psubarg#1#2{\P_{#1}\left[{#2}\right]}
\def\Cov{\mrm{Cov}} % Covariance symbol
\def\Covarg#1{\Cov\left[{#1}\right]}
\def\Covsubarg#1#2{\Cov_{#1}\left[{#2}\right]}
\def\Var{\mrm{Var}}
\def\Vararg#1{\Var\left(#1\right)}
\def\Varsubarg#1#2{\Var_{#1}\left(#2\right)}
\newcommand{\family}{\mathcal{P}} % probability family
\newcommand{\eps}{\epsilon}
\def\absarg#1{\left|#1\right|}
\def\msarg#1{\left(#1\right)^{2}}
\def\logarg#1{\log\left(#1\right)}
%~~~~~~~~~~~~~~~
% Distributions
%~~~~~~~~~~~~~~~
\def\Gsn{\mathcal{N}}
\def\Ber{\textnormal{Ber}}
\def\Bin{\textnormal{Bin}}
\def\Unif{\textnormal{Unif}}
\def\Mult{\textnormal{Mult}}
\def\Cat{\textnormal{Cat}}
\def\Gam{\textnormal{Gam}}
\def\InvGam{\textnormal{InvGam}}
\def\NegMult{\textnormal{NegMult}}
\def\Dir{\textnormal{Dir}}
\def\Lap{\textnormal{Laplace}}
\def\Bet{\textnormal{Beta}}
\def\Poi{\textnormal{Poi}}
\def\HypGeo{\textnormal{HypGeo}}
\def\GEM{\textnormal{GEM}}
\def\BP{\textnormal{BP}}
\def\DP{\textnormal{DP}}
\def\BeP{\textnormal{BeP}}
%~~~~~~~~~~~~~~~
% Theorem-like environments
%~~~~~~~~~~~~~~~

%-----------------------
% Probability sets
%-----------------------
\newcommand{\X}{\mathcal{X}}
\newcommand{\Y}{\mathcal{Y}}
\newcommand{\D}{\mathcal{D}}
\newcommand{\Scal}{\mathcal{S}}
%-----------------------
% vector notation
%-----------------------
\newcommand{\bx}{\mathbf{x}}
\newcommand{\by}{\mathbf{y}}
\newcommand{\bt}{\mathbf{t}}
\newcommand{\xbar}{\overline{x}}
\newcommand{\Xbar}{\overline{X}}
\newcommand{\tolaw}{\xrightarrow{\mathcal{L}}}
\newcommand{\toprob}{\xrightarrow{\mathbb{P}}}
\newcommand{\laweq}{\overset{\mathcal{L}}{=}}
\newcommand{\F}{\mathcal{F}}
\def\colarg#1#2{\textcolor[HTML]{#1}{#2}}

\setbeamersize{text margin left=5pt,text margin right=5pt}

\setbeamerfont{institute}{size=\fontsize{7pt}{8pt}}
\setbeamerfont{date}{size=\fontsize{0pt}{0pt}}

\title{Methods for Regime Detection and Multitable Analysis}
\author{Kris Sankaran}

\begin{document}

\begin{frame}
  \frametitle{Problems of Interest}
 \begin{itemize}
 \item Regime Detection: Can we segment time intervals in a microbiome study in
   an unsupervised way?
   \begin{itemize}
   \item Intervals with similar dynamics should be placed in the same regime
   \item Can we characterize differences between taxa during these regimes?
   \end{itemize}
 \item Multitable Analysis: How should we simultaneously study multiple
   measurement types?
   \begin{itemize}
   \item How do different measurement types covary with one another?
   \item Compared to usual regression, when there are multiple related
     responses, there are fewer obvious practical approaches
   \end{itemize}
 \end{itemize} 
\end{frame}

\begin{frame}
  \frametitle{Goals}
  \begin{itemize}
  \item Distill relevant themes across analysis approaches
  \item Provide concrete workflows for getting started with an analysis
  \item Ultimately, provide thoughtful recommendations tailored to specific
    study aims and data characteristics
  \end{itemize} 
\end{frame}

\section{Regime Detection}
\label{sec:regime_detection}

\begin{frame}
  \frametitle{Antibiotics Study}
 \begin{itemize}
 \item 
 \end{itemize} 
\end{frame}

\section{Multitable Analysis}
\label{sec:multitable analysis}

\begin{frame}
  \frametitle{WELL-China Study}
  \begin{itemize}
  \item Focused on the relationship between various indicators of wellness.
  \item Data about exercise, sleep, diet, mental health for 1969 people, in
    addition to 16s data on 221 participants (with plans for metabolomics,
    methylation, ...)
  \item We focus on question of body composition vs. microbiome structure
  \end{itemize}
\end{frame}

\begin{frame}
  \frametitle{Body Composition}
  \begin{itemize}
  \item Scans from DXA sensor to provide richer information than BMI alone
  \item Can we reproduce findings relating microbiome and BMI / metabolic
    syndrome, but using our richer data on body type?
    \begin{itemize}
    \item Relevant in light of the fact that people with similar BMI might have
      very different disease dispositions depending on the distribution of lean
      and fat mass
    \end{itemize}
  \end{itemize}  
\end{frame}

\begin{frame}
  \frametitle{Conclusion}
  
\end{frame}

\section{Testing using Distances}
\label{sec:testing_distances}

\end{document}
