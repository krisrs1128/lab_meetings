\documentclass{beamer}
\usetheme{Madrid}
\usepackage{natbib}
\usepackage{sidecap}
\usepackage{hyperref}
\usepackage{graphicx}
% ------------------------------------------------------------------------
% Packages
% ------------------------------------------------------------------------
\usepackage{amsmath}

% ------------------------------------------------------------------------
% Macros
% ------------------------------------------------------------------------
%~~~~~~~~~~~~~~~
% List shorthand
%~~~~~~~~~~~~~~~
\newcommand{\BIT}{\begin{itemize}}
\newcommand{\EIT}{\end{itemize}}
\newcommand{\BNUM}{\begin{enumerate}}
\newcommand{\ENUM}{\end{enumerate}}
%~~~~~~~~~~~~~~~
% Text with quads around it
%~~~~~~~~~~~~~~~
\newcommand{\qtext}[1]{\quad\text{#1}\quad}
%~~~~~~~~~~~~~~~
% Shorthand for math formatting
%~~~~~~~~~~~~~~~
\newcommand\mbb[1]{\mathbb{#1}}
\newcommand\mbf[1]{\mathbf{#1}}
\def\mc#1{\mathcal{#1}}
\def\mrm#1{\mathrm{#1}}
%~~~~~~~~~~~~~~~
% Common sets
%~~~~~~~~~~~~~~~
\def\reals{\mathbb{R}} % Real number symbol
\def\integers{\mathbb{Z}} % Integer symbol
\def\rationals{\mathbb{Q}} % Rational numbers
\def\naturals{\mathbb{N}} % Natural numbers
\def\complex{\mathbb{C}} % Complex numbers
\def\simplex{\mathcal{S}} % Simplex
%~~~~~~~~~~~~~~~
% Common functions
%~~~~~~~~~~~~~~~
\renewcommand{\exp}[1]{\operatorname{exp}\left(#1\right)} % Exponential
\def\indic#1{\mbb{I}\left({#1}\right)} % Indicator function
\providecommand{\argmax}{\mathop\mathrm{arg max}} % Defining math symbols
\providecommand{\argmin}{\mathop\mathrm{arg min}}
\providecommand{\arccos}{\mathop\mathrm{arccos}}
\providecommand{\asinh}{\mathop\mathrm{asinh}}
\providecommand{\dom}{\mathop\mathrm{dom}} % Domain
\providecommand{\range}{\mathop\mathrm{range}} % Range
\providecommand{\diag}{\mathop\mathrm{diag}}
\providecommand{\tr}{\mathop\mathrm{tr}}
\providecommand{\abs}{\mathop\mathrm{abs}}
\providecommand{\card}{\mathop\mathrm{card}}
\providecommand{\sign}{\mathop\mathrm{sign}}
\def\rank#1{\mathrm{rank}({#1})}
\def\supp#1{\mathrm{supp}({#1})}
%~~~~~~~~~~~~~~~
% Common probability symbols
%~~~~~~~~~~~~~~~
\def\E{\mathbb{E}} % Expectation symbol
\def\Earg#1{\E\left[{#1}\right]}
\def\Esubarg#1#2{\E_{#1}\left[{#2}\right]}
\def\P{\mathbb{P}} % Probability symbol
\def\Parg#1{\P\left({#1}\right)}
\def\Psubarg#1#2{\P_{#1}\left[{#2}\right]}
\def\Cov{\mrm{Cov}} % Covariance symbol
\def\Covarg#1{\Cov\left[{#1}\right]}
\def\Covsubarg#1#2{\Cov_{#1}\left[{#2}\right]}
\def\Var{\mrm{Var}}
\def\Vararg#1{\Var\left(#1\right)}
\def\Varsubarg#1#2{\Var_{#1}\left(#2\right)}
\newcommand{\family}{\mathcal{P}} % probability family
\newcommand{\eps}{\epsilon}
\def\absarg#1{\left|#1\right|}
\def\msarg#1{\left(#1\right)^{2}}
\def\logarg#1{\log\left(#1\right)}
%~~~~~~~~~~~~~~~
% Distributions
%~~~~~~~~~~~~~~~
\def\Gsn{\mathcal{N}}
\def\Ber{\textnormal{Ber}}
\def\Bin{\textnormal{Bin}}
\def\Unif{\textnormal{Unif}}
\def\Mult{\textnormal{Mult}}
\def\Cat{\textnormal{Cat}}
\def\Gam{\textnormal{Gam}}
\def\InvGam{\textnormal{InvGam}}
\def\NegMult{\textnormal{NegMult}}
\def\Dir{\textnormal{Dir}}
\def\Lap{\textnormal{Laplace}}
\def\Bet{\textnormal{Beta}}
\def\Poi{\textnormal{Poi}}
\def\HypGeo{\textnormal{HypGeo}}
\def\GEM{\textnormal{GEM}}
\def\BP{\textnormal{BP}}
\def\DP{\textnormal{DP}}
\def\BeP{\textnormal{BeP}}
%~~~~~~~~~~~~~~~
% Theorem-like environments
%~~~~~~~~~~~~~~~

%-----------------------
% Probability sets
%-----------------------
\newcommand{\X}{\mathcal{X}}
\newcommand{\Y}{\mathcal{Y}}
\newcommand{\D}{\mathcal{D}}
\newcommand{\Scal}{\mathcal{S}}
%-----------------------
% vector notation
%-----------------------
\newcommand{\bx}{\mathbf{x}}
\newcommand{\by}{\mathbf{y}}
\newcommand{\bt}{\mathbf{t}}
\newcommand{\xbar}{\overline{x}}
\newcommand{\Xbar}{\overline{X}}
\newcommand{\tolaw}{\xrightarrow{\mathcal{L}}}
\newcommand{\toprob}{\xrightarrow{\mathbb{P}}}
\newcommand{\laweq}{\overset{\mathcal{L}}{=}}
\newcommand{\F}{\mathcal{F}}
\def\colarg#1#2{\textcolor[HTML]{#1}{#2}}


\title{Visualization and Modeling of Switching Dynamics}
\author{Kris Sankaran}

\begin{document}

\begin{frame}
  \titlepage
\end{frame}

\begin{frame}
  \frametitle{Outline}
\begin{itemize}
\item \textbf{Problem Description} [5 min]: Can we automatically identify
  different dynamic regimes among different bacteria?
\item \textbf{Heuristic approaches} [10 min]: How would we approach this problem
  using standard techniques?
\item \textbf{Visualization experiments} [10 min]: How can we leverage linking
  and focus + context?
\item \textbf{Review of relevant models} [30 min]: HMMs, HDP-HMMs, Switching
  LDS, and Dynamic Tobit
\end{itemize}
\end{frame}

\begin{frame}
  \frametitle{Problem Description}
  \begin{itemize}
  \item Our goal is a succinct representation of which bacteria have which types
    of responses to different perturbations
    \begin{itemize}
    \item If two types of perturbation lead to similar dynamics, we should be able
      to identify that
    \item Similarly, we should be able to see whether different groups of bacteria
      have similar responses to a perturbation
    \end{itemize} 
  \item We will study the problem from several angles, though ultimately the
    goal is to have a more unified approach (or at least guidelines)
  \end{itemize}  
\end{frame}

\begin{frame}
  \frametitle{Heuristic Approaches} 
  \begin{itemize}
  \item Ignoring temporal structure, a straightforward approach is to simply
    cluster bacteria
  \item The centroids at different subtrees can give useful interpretations (what
    taxonomic groups have which time series shapes?)
  \item There are a few choices to make,
    \begin{itemize}
    \item How should we transform the data?
    \item What distance should we use?
    \end{itemize} 
  \end{itemize}
\end{frame}

\begin{frame}
  \frametitle{What transformation?}
 \begin{figure}
   \centering
   \includegraphics{figure/heatmap-euclidean}
   \caption{A first approach is to cluster the asinh transformed data using an
     ordinary euclidean distance. Here, each row represents an RSV, each column
     is a timepoint for one of the three subjects (the three facet rows).
     Columns are sorted according to their position on the hierarchical
     clustering dendrogram, with neighboring subtrees split into separate
     facets. \label{fig:heatmap-euclidean}}
 \end{figure}
\end{frame}

\begin{frame}
  \frametitle{Alternatively, cluster the innovations}
 \begin{figure}[ht]
   \centering
   \includegraphics{figure/heatmap-innovations}
   \caption{To capture one aspect of temporal dynamics, we can cluster the
     differenced data instead. However, if two series return to different
     levels, and are steady after that point, we would not be able to
     distinguish the difference. \label{fig:heatmap-innovations} }
 \end{figure}
\end{frame}

\begin{frame}
  \frametitle{Jaccard distance}
 \begin{figure}[ht]
   \centering
   \includegraphics{figure/heatmap-jaccard}
   \caption{To account for the presence of zeros, we could use a Jaccard
     distance, though this loses information on the abundance when present.
     \label{fig:heatmap-jaccard} }
 \end{figure}
\end{frame}

\begin{frame}
  \frametitle{Combining distances}
 \begin{figure}[ht]
   \centering
   \includegraphics{figure/heatmap-mix}
   \caption{As a compromise, we can cluster based on a mixture distance, $d_{ij}
     = \lambda d^{(1)}_{ij} + \left(1 - \lambda\right)d^{(2)}_{ij}$, though the
    effect of different choices of $\lambda$ is not entirely clear.
     \label{fig:heatmap-jaccard} }
 \end{figure}
\end{frame}

\begin{frame}
  \frametitle{Interpreting centroids}
\begin{figure}[ht]
  \centering
  \includegraphics[width=\textwidth]{figure/centroid-mix-conditional}
  \caption{A reduced summary of each cluster is available by inspecting
    centroids. \label{fig:centroid-mix-conditional} }
\end{figure}
\end{frame}

\begin{frame}
  \frametitle{Interpreting centroids (presence vs. absence)}
\begin{figure}[ht]
  \centering
  \includegraphics[width=\textwidth]{figure/centroid-mix-presence}
  \caption{We can also look at the fraction of RSVs that have nonzero abundance
    at each timepoint, within each cluster. \label{fig:centroid-mix-presence} }
\end{figure}
\end{frame}

\begin{frame}
  \frametitle{Visualization Motivation}
  \begin{itemize}
  \item On their own, these visualizations don’t provide much of a reduction of the data
  \item Comparing centroids corresponding to different subtrees is difficult
    \begin{itemize}
    \item Even relating centroids from subtrees obtained by cutting at two heights is complicated
    \item This visual navigation question can be addressed using linking and focus + context
    \end{itemize}
  \end{itemize}
\end{frame}

\begin{frame}
  \frametitle{Visualization Approach}
  \begin{itemize}
  \item \href{/Users/krissankaran/Desktop/100_days/july3/hclust.html}{link}
  \item Link heatmap, tree, and centroids: A box is drawn across bacteria descending from a node, and the centroid for those series is shown at the top right
  \item Each time series corresponds to an individual
  \item The breakdown of the subtree into taxonomic families is shown in the bottom right
  \item Different clusters can be compared side by side
  \end{itemize} 
\end{frame}

\begin{frame}
  \frametitle{Hidden Markov Models (HMMs)}
  \begin{itemize}
  \item To incorporate dynamics more explicitly, we can use HMMs
  \item We suppose that each timepoints belongs to one of K states, that these
    states evolve according to a markov model, and that the observed data are iid
    draws from an emission distribution associated with the given state
  \end{itemize}
\end{frame}

\begin{frame}
  \frametitle{HMMs}
\begin{itemize}
\item Let $\left(y_{t}^{i}\right)$ be a collection of observation sequences
  (each in $\reals^{p}$) and $\left(z_{t}^{i}\right)$ be the associated states
  (in $\{1, \dots, K\}$)
  \begin{itemize}
  \item Note that multiple bacteria are all being drawn from the same K states
  \end{itemize}
\item The complete data likelihood is
  \begin{align*}
    p\left(\left(y_{t}^{i}\right), \left(z_{t}^{i}\right)\right) &= \prod_{i = 1}^{n} \left[\prod_{t = 1}^{T} p\left(y_{t}^{i} \vert z_{t}^{i}\right) p\left(z_{0}\right)\prod_{t = 2}^{T} p\left(z_{t} \vert z_{t - 1}\right)\right]
  \end{align*}
\item This can be fit using the EM algorithm, where we optimize the expectation
  of the complete data loglikelihood over (1) the distribution on latent states
  and (2) parameters of the Markov chain and emission distributions
\end{itemize}  
\end{frame}

\bibliographystyle{plainnat}
\bibliography{refs.bib}
 
\end{document}
