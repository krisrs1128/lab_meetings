\documentclass{beamer}
\usetheme{Madrid}
\usepackage{natbib}
\usepackage{sidecap}
\usepackage{hyperref}
\usepackage{verbatim}
\usepackage{subfig}
\usepackage{graphicx}
% ------------------------------------------------------------------------
% Packages
% ------------------------------------------------------------------------
\usepackage{amsmath}

% ------------------------------------------------------------------------
% Macros
% ------------------------------------------------------------------------
%~~~~~~~~~~~~~~~
% List shorthand
%~~~~~~~~~~~~~~~
\newcommand{\BIT}{\begin{itemize}}
\newcommand{\EIT}{\end{itemize}}
\newcommand{\BNUM}{\begin{enumerate}}
\newcommand{\ENUM}{\end{enumerate}}
%~~~~~~~~~~~~~~~
% Text with quads around it
%~~~~~~~~~~~~~~~
\newcommand{\qtext}[1]{\quad\text{#1}\quad}
%~~~~~~~~~~~~~~~
% Shorthand for math formatting
%~~~~~~~~~~~~~~~
\newcommand\mbb[1]{\mathbb{#1}}
\newcommand\mbf[1]{\mathbf{#1}}
\def\mc#1{\mathcal{#1}}
\def\mrm#1{\mathrm{#1}}
%~~~~~~~~~~~~~~~
% Common sets
%~~~~~~~~~~~~~~~
\def\reals{\mathbb{R}} % Real number symbol
\def\integers{\mathbb{Z}} % Integer symbol
\def\rationals{\mathbb{Q}} % Rational numbers
\def\naturals{\mathbb{N}} % Natural numbers
\def\complex{\mathbb{C}} % Complex numbers
\def\simplex{\mathcal{S}} % Simplex
%~~~~~~~~~~~~~~~
% Common functions
%~~~~~~~~~~~~~~~
\renewcommand{\exp}[1]{\operatorname{exp}\left(#1\right)} % Exponential
\def\indic#1{\mbb{I}\left({#1}\right)} % Indicator function
\providecommand{\argmax}{\mathop\mathrm{arg max}} % Defining math symbols
\providecommand{\argmin}{\mathop\mathrm{arg min}}
\providecommand{\arccos}{\mathop\mathrm{arccos}}
\providecommand{\asinh}{\mathop\mathrm{asinh}}
\providecommand{\dom}{\mathop\mathrm{dom}} % Domain
\providecommand{\range}{\mathop\mathrm{range}} % Range
\providecommand{\diag}{\mathop\mathrm{diag}}
\providecommand{\tr}{\mathop\mathrm{tr}}
\providecommand{\abs}{\mathop\mathrm{abs}}
\providecommand{\card}{\mathop\mathrm{card}}
\providecommand{\sign}{\mathop\mathrm{sign}}
\def\rank#1{\mathrm{rank}({#1})}
\def\supp#1{\mathrm{supp}({#1})}
%~~~~~~~~~~~~~~~
% Common probability symbols
%~~~~~~~~~~~~~~~
\def\E{\mathbb{E}} % Expectation symbol
\def\Earg#1{\E\left[{#1}\right]}
\def\Esubarg#1#2{\E_{#1}\left[{#2}\right]}
\def\P{\mathbb{P}} % Probability symbol
\def\Parg#1{\P\left({#1}\right)}
\def\Psubarg#1#2{\P_{#1}\left[{#2}\right]}
\def\Cov{\mrm{Cov}} % Covariance symbol
\def\Covarg#1{\Cov\left[{#1}\right]}
\def\Covsubarg#1#2{\Cov_{#1}\left[{#2}\right]}
\def\Var{\mrm{Var}}
\def\Vararg#1{\Var\left(#1\right)}
\def\Varsubarg#1#2{\Var_{#1}\left(#2\right)}
\newcommand{\family}{\mathcal{P}} % probability family
\newcommand{\eps}{\epsilon}
\def\absarg#1{\left|#1\right|}
\def\msarg#1{\left(#1\right)^{2}}
\def\logarg#1{\log\left(#1\right)}
%~~~~~~~~~~~~~~~
% Distributions
%~~~~~~~~~~~~~~~
\def\Gsn{\mathcal{N}}
\def\Ber{\textnormal{Ber}}
\def\Bin{\textnormal{Bin}}
\def\Unif{\textnormal{Unif}}
\def\Mult{\textnormal{Mult}}
\def\Cat{\textnormal{Cat}}
\def\Gam{\textnormal{Gam}}
\def\InvGam{\textnormal{InvGam}}
\def\NegMult{\textnormal{NegMult}}
\def\Dir{\textnormal{Dir}}
\def\Lap{\textnormal{Laplace}}
\def\Bet{\textnormal{Beta}}
\def\Poi{\textnormal{Poi}}
\def\HypGeo{\textnormal{HypGeo}}
\def\GEM{\textnormal{GEM}}
\def\BP{\textnormal{BP}}
\def\DP{\textnormal{DP}}
\def\BeP{\textnormal{BeP}}
%~~~~~~~~~~~~~~~
% Theorem-like environments
%~~~~~~~~~~~~~~~

%-----------------------
% Probability sets
%-----------------------
\newcommand{\X}{\mathcal{X}}
\newcommand{\Y}{\mathcal{Y}}
\newcommand{\D}{\mathcal{D}}
\newcommand{\Scal}{\mathcal{S}}
%-----------------------
% vector notation
%-----------------------
\newcommand{\bx}{\mathbf{x}}
\newcommand{\by}{\mathbf{y}}
\newcommand{\bt}{\mathbf{t}}
\newcommand{\xbar}{\overline{x}}
\newcommand{\Xbar}{\overline{X}}
\newcommand{\tolaw}{\xrightarrow{\mathcal{L}}}
\newcommand{\toprob}{\xrightarrow{\mathbb{P}}}
\newcommand{\laweq}{\overset{\mathcal{L}}{=}}
\newcommand{\F}{\mathcal{F}}
\def\colarg#1#2{\textcolor[HTML]{#1}{#2}}


\title{Visualization and Modeling of Switching Dynamics}
\author{Kris Sankaran}

\begin{document}

\begin{frame}
  \titlepage
\end{frame}

\begin{frame}
  \frametitle{Outline}
\begin{itemize}
\item \textbf{Problem Description} [5 min]: Can we automatically identify
  different dynamic regimes among different bacteria?
\item \textbf{Heuristic approaches} [10 min]: How would we approach this problem
  using standard techniques?
\item \textbf{Visualization experiments} [10 min]: How can we leverage linking
  and focus + context?
\item \textbf{Review of relevant models} [30 min]: HMMs, HDP-HMMs, Switching
  LDS, and Dynamic Tobit
\end{itemize}
\end{frame}

\begin{frame}
  \frametitle{Problem Description}
  \begin{itemize}
  \item Our goal is a succinct representation of which bacteria have which types
    of responses to different perturbations
    \begin{itemize}
    \item If two types of perturbation lead to similar dynamics, we should be able
      to identify that
    \item Similarly, we should be able to see whether different groups of bacteria
      have similar responses to a perturbation
    \end{itemize} 
  \item We will study the problem from several angles, though ultimately the
    goal is to have a more unified approach (or at least guidelines)
  \end{itemize}  
\end{frame}

\begin{frame}
  \frametitle{Heuristic Approaches} 
  \begin{itemize}
  \item Ignoring temporal structure, a straightforward approach is to simply
    cluster bacteria
  \item The centroids at different subtrees can give useful interpretations (what
    taxonomic groups have which time series shapes?)
  \item There are a few choices to make,
    \begin{itemize}
    \item How should we transform the data?
    \item What distance should we use?
    \end{itemize} 
  \end{itemize}
\end{frame}

\begin{frame}
  \frametitle{What transformation?}
 \begin{figure}
   \centering
   \includegraphics{figure/heatmap-euclidean}
   \caption{A first approach is to cluster the asinh transformed data using an
     ordinary euclidean distance. Here, each row represents an RSV, each column
     is a timepoint for one of the three subjects (the three facet columns).
     Rows are sorted according to their position on the hierarchical
     clustering dendrogram, with neighboring subtrees split into separate
     facets. \label{fig:heatmap-euclidean}}
 \end{figure}
\end{frame}

\begin{frame}
  \frametitle{Alternatively, cluster the innovations}
 \begin{figure}[ht]
   \centering
   \includegraphics{figure/heatmap-innovations}
   \caption{To capture one aspect of temporal dynamics, we can cluster the
     differenced data instead. However, if two series return to different
     levels, and are steady after that point, we would not be able to
     distinguish them. \label{fig:heatmap-innovations} }
 \end{figure}
\end{frame}

\begin{frame}
  \frametitle{Jaccard distance}
 \begin{figure}[ht]
   \centering
   \includegraphics{figure/heatmap-jaccard}
   \caption{To account for the presence of zeros, we could use a Jaccard
     distance, though this loses information on all nonzero abundances.
     \label{fig:heatmap-jaccard} }
 \end{figure}
\end{frame}

\begin{frame}
  \frametitle{Combining distances}
 \begin{figure}[ht]
   \centering
   \includegraphics{figure/heatmap-mix}
   \caption{As a compromise, we can cluster based on a mixture distance, $d_{ij}
     = \lambda d^{(1)}_{ij} + \left(1 - \lambda\right)d^{(2)}_{ij}$, though the
    effect of different choices of $\lambda$ is not entirely clear.
     \label{fig:heatmap-jaccard} }
 \end{figure}
\end{frame}

\begin{frame}
  \frametitle{Interpreting centroids}
\begin{figure}[ht]
  \centering
  \includegraphics[width=\textwidth]{figure/centroid-mix-conditional}
  \caption{A reduced summary of each cluster is available by inspecting
    centroids. \label{fig:centroid-mix-conditional} }
\end{figure}
\end{frame}

\begin{frame}
  \frametitle{Interpreting centroids (presence vs. absence)}
\begin{figure}[ht]
  \centering
  \includegraphics[width=\textwidth]{figure/centroid-mix-presence}
  \caption{We can also look at the fraction of RSVs that have nonzero abundance
    at each timepoint, within each cluster. \label{fig:centroid-mix-presence} }
\end{figure}
\end{frame}

\begin{frame}
  \frametitle{Visualization Motivation}
  \begin{itemize}
  \item On their own, these visualizations don't provide much of a reduction of the data
  \item Comparing centroids corresponding to different subtrees is difficult
    \begin{itemize}
    \item Even relating centroids from subtrees obtained by cutting at two heights is complicated
    \item This visual navigation question can be addressed using linking and focus + context
    \end{itemize}
  \end{itemize}
\end{frame}

\begin{frame}
  \frametitle{Visualization Approach}
  \begin{itemize}
  \item \href{/Users/krissankaran/Desktop/100_days/july3/hclust.html}{link}
  \item Link heatmap, tree, and centroids: A box is drawn across bacteria descending from a node, and the centroid for those series is shown at the top right
  \item Each time series corresponds to an individual
  \item The breakdown of the subtree into taxonomic families is shown in the bottom right
  \item Different clusters can be compared side by side
  \end{itemize} 
\end{frame}

\begin{frame}
  \frametitle{Hidden Markov Models (HMMs)}
  \begin{itemize}
  \item To incorporate dynamics more explicitly, we can use HMMs
  \item We suppose that each timepoints belongs to one of $K$ states, that these
    states evolve according to a markov model, and that the observed data are iid
    draws from an emission distribution associated with the given state
  \end{itemize}
\end{frame}

\begin{frame}
  \frametitle{HMMs}
\begin{itemize}
\item Let $\left(y_{t}^{i}\right)$ be a collection of observation sequences
  (each in $\reals^{p}$) and $\left(z_{t}^{i}\right)$ be the associated states
  (in $\{1, \dots, K\}$)
  \begin{itemize}
  \item Note that multiple bacteria are all being drawn from the same K states
  \end{itemize}
\item The complete data likelihood is
  \begin{align*}
    p\left(\left(y_{t}^{i}\right), \left(z_{t}^{i}\right)\right) &= \prod_{i = 1}^{n} \left[\prod_{t = 1}^{T} p\left(y_{t}^{i} \vert z_{t}^{i}\right) p\left(z_{0}\right)\prod_{t = 2}^{T} p\left(z_{t} \vert z_{t - 1}\right)\right]
  \end{align*}
\item This can be fit using the EM algorithm, where we optimize the expectation
  of the complete data loglikelihood over (1) the distribution on latent states
  and (2) parameters of the Markov chain and emission distributions
\end{itemize}  
\end{frame}

\begin{frame}
  \frametitle{Smoothed States}
  \begin{figure}[ht]
  \centering
  \includegraphics[width = 0.37\textwidth]{figure/hmm_mode}
  \caption{We plot the states with highest smoothing probability
    $p\left(z_{t}^{i)} \vert \left(y_{t}^{i}\right)\right)$ for each cell.
    \label{fig:hmm_mode} }
\end{figure}
\end{frame}

\begin{frame}
  \frametitle{Estimated transitions}
  \begin{itemize}
    \item The antibiotic effect is visible in the elevated probabilities of
      transitioning between the two lowest states
      \item Columns and rows are sorted from highest to lowest emission mean
\begin{verbatim} 
0.910 0.056 0.007 0.027 \\
0.019 0.810 0.094 0.077 \\
0.002 0.083 0.527 0.389 \\
0.003 0.028 0.159 0.810 
\end{verbatim}
  \end{itemize}
\end{frame}

\begin{frame}
  \frametitle{Sticky HMMs}
\begin{itemize}
\item While not necessarily relevant to our problem, it is useful to be aware
of approaches that induce ``stickiness''
\item We can place a prior on the transition probabilities that encourages
  self-transitions
  \item Formally, draw the $k^{th}$ row of the transition probability matrix as
\begin{align*}
  \pi_{k} \sim \Dir\left(\alpha_{1}, \dots, \alpha_{k} + \kappa, \dots, \alpha_{K}\right),
\end{align*}
where $\kappa$ controls the preference of a state to stay where it is
\item The EM algorithm is longer tractable, but we can use (blocked) gibbs
  sampling -- sequentially sample all the $z_{t}^{i}$s, then the emission
  parameters, then the transition probabilities
\end{itemize}
\end{frame}

\begin{frame}
  \frametitle{Simulation}
  \begin{figure}
    \centering
    \subfloat[Truth]{{\includegraphics[width=5cm]{figure/sticky_hmm_truth} }}
    \qquad
    \subfloat[Observed]{{\includegraphics[width=5cm]{figure/sticky_hmm_observed} }}
    \caption{Each row corresponds to a time series. Different colors correspond
      to different true underlying states.}
    \label{fig:sticky_hmm_kappas}
  \end{figure}

\end{frame}

\begin{frame}
  \frametitle{Simulation}
  \begin{figure}
    \centering
    \subfloat[$\kappa = 0$]{{\includegraphics[width=5cm]{figure/sticky_hmm_no_kappa} }}%
    \qquad
    \subfloat[$\kappa = 3$]{{\includegraphics[width=5cm]{figure/sticky_hmm_large_kappa} }}%
    \caption{The effect of the $\kappa$ parameter on the sampled $z$'s, at a
      single gibbs sampling iteration.}
    \label{fig:sticky_hmm_kappas}
  \end{figure}
\end{frame}

\begin{frame}
  \frametitle{Antibiotics data}
  \begin{figure}[ht]
    \centering
    \includegraphics[width = 0.37\textwidth]{figure/bayes_mode}
    \caption{This is the analog of Figure \ref{fig:hmm_mode}, now for the sticky
      HMM with $\kappa = 4$. \label{fig:hmm_mode} }
  \end{figure}
\end{frame}

\begin{frame}
  \frametitle{Estimated transitions}
  \begin{itemize}
    \item Apparently, even though we have introduced a prior encouraging
      self-transitions, the fact that we have a prior at all regularizes
      probabilities away from 0 and 1.
\begin{verbatim} 
0.775 (0.004) 0.113 (0.005) 0.044 (0.002) 0.068 (0.004) \\
0.160 (0.009) 0.414 (0.016) 0.170 (0.010) 0.256 (0.016) \\ 
0.074 (0.008) 0.300 (0.013) 0.195 (0.029) 0.431 (0.012) \\
0.025 (0.002) 0.083 (0.006) 0.083 (0.005) 0.809 (0.004) 
\end{verbatim}
  \end{itemize}
\end{frame}

\begin{frame}
  \frametitle{Sampled $z$'s}
\begin{figure}[ht]
  \centering
  \includegraphics[width=0.5\textwidth]{figure/gibbs_samples}
  \caption{We can directly inspect samples from the $z$ sequences. Evidently,
    the states with larger means mix reasonably well, but not the others --
    this may be a consequence of the presence of many exact
    zeros. \label{fig:gibbs_samples} }
\end{figure}
\end{frame}

\begin{frame}
  \frametitle{Sticky HDP-HMMs}
 \begin{itemize}
 \item We might want to introduce many states, since there are many bacteria,
   but we might only want to have a few states per bacteria
 \item This can be accomplished heuristically using an HDP-HMM
   \citep{fox2007sticky}, and more carefully using a beta-process sharing
   framework \citep{fox2009sharing}
 \end{itemize} 
\end{frame}

\begin{frame}
  \frametitle{Simulation}
  \begin{figure}
    \centering
    \subfloat[Sticky HMM]{{\includegraphics[width=5cm]{figure/sticky_hmm_large_kappa} }}%
    \qquad
    \subfloat[Sticky HDP-HMM]{{\includegraphics[width=5cm]{figure/sticky_hdp_hmm} }}%
    \caption{The sticky HDP-HMM introduces more states, but doesn't necessarily
      use them all in each sequence.}
    \label{fig:sticky_hdp_hmm}
  \end{figure}
\end{frame}

\begin{frame}
  \frametitle{Antibiotics data}
  \begin{figure}[ht]
    \centering
    \includegraphics[width = 0.37\textwidth]{figure/hdp_antibiotics_mode}
    \caption{This is the analog of Figure \ref{fig:hmm_mode}, now for the sticky
      HDP-HMM. We find mixing behavior to be similar to the original sticky
      HMM. \label{fig:hdp_mode} }
  \end{figure}
\end{frame}

\begin{frame}
  \frametitle{Switching State Space Models (SSMs)}
 \begin{itemize}
 \item A limitation of HMMs is that they assume IID behavior within states
 \item In the real data however, we observe differences in dynamics
 \item Switching SSMs provide a framework for modeling this \citep{ghahramani2000variational}
 \end{itemize} 

\begin{figure}[ht]
  \centering
  \includegraphics[width=0.6\textwidth]{figure/ssm-motivation}
  \caption{The difference in recovery between these two clusters reflects more
    complex structure than simply switching
    emissions. \label{fig:ssm-motivation}}
\end{figure}

\end{frame}

\begin{frame}
  \frametitle{Model description}
\begin{itemize}
\item Underlying time series modes switch according to a Markov model
\item Within a given mode, the observed sequence is drawn according to one of
  the $K$ state space models
  \begin{align*}
    z_{t} \vert z_{t - 1} &\sim \Mult\left(1, \pi_{z_{t - 1}}\right) \\
    x^{k}_{t} &= A_{k}x_{t - 1}^{k} + w^{k}_{t} \text{ for }k = 1, \dots, K\\
    y_{t} &= C_{z_{t}}x_{t}^{k} + \eps_{t}^{z_{t}}
    \end{align*}
  where the noise is distributed according to
  \begin{align*}
    w^{k}_{t} &\sim \Gsn\left(0, Q_{k}\right) \\
    \eps^{k}_{t} &\sim \Gsn\left(0, R_{k}\right)
  \end{align*}
\end{itemize}
\end{frame}

\begin{frame}
  \frametitle{Model Estimation}
  \begin{itemize}
  \item Heuristic, variational, MCMC, and particle filtering estimation
    techniques have all been studied \citep{harrison1976bayesian, ghahramani2000variational, fruhwirth2001markov, doucet2001particle}
  \item We will show results from a variational formulation
  \item We alternately (1) run $K$ weighted EM algorithms for state space
    estimation (Kalman filtering + M step), according to posterior probabilities
    for state membership, and (2) run an HMM to infer the underlying state
    membership
  \item Annealing and good initialization is necessary to avoid bad local minima
  \end{itemize}
\end{frame}

\begin{frame}
  \frametitle{Simulation example}
\begin{figure}[ht]
  \centering
  \includegraphics[width=0.8\textwidth]{figure/ssm_truth}
  \caption{The two halves of this sequence are drawn from two very different
    dynamic regimes, one with low noise and high autocorrelation, and one with
    high noise and low autocorrelation. The gray line represents the true
    underlying $x^{z_{t}}$. \label{fig:ssm_truth} }
\end{figure}
\end{frame}

\begin{frame}
  \frametitle{Simulation example}
\begin{figure}[ht]
  \centering
  \includegraphics[width=0.8\textwidth]{figure/ssm_fit}
  \caption{The two new lines represent two of the fitted $x_{t}^{k}$ sequences.
    The color indicates the probability $\Parg{z_{t} = 1 \vert y_{t}}$,
      according to the variational approximation. The first few points are
      incorrectly placed in the high-noise regime.\label{fig:ssm_truth} }
\end{figure}
\end{frame}

\begin{frame}
  \frametitle{Dynamic Tobit model}
 \begin{itemize}
 \item Our emission structure is not really plausible for a number of bacteria
 \item We could try to work with the untransformed count data, and use a
   discrete distribution for the emissions
  \item Alternatively, the Dynamic Tobit model adds an extra layer of censoring
    on top of the ordinary state space model,
    \begin{align*}
      x_{t} &= Ax_{t - 1} + w_{t} \\
      \tilde{y}_{t} &= Cx_{t} + \eps_{t} \\
      y_{t} &= \tilde{y}_{t} \indic{\tilde{y}_{t} \geq c},
    \end{align*}
    where both $x$ and $\tilde{y}$ are unobserved.
 \end{itemize} 
\begin{figure}[ht]
  \centering
  \includegraphics[width=0.6\textwidth]{figure/tobit-motivation}
  \caption{Many series appear to have brief excursions above 0 \label{fig:label}}
\end{figure}
\end{frame}

\begin{frame}
  \frametitle{Dynamic Tobit model}
  \begin{itemize}
  \item Estimation of these models has been studied in \citep{chib1992bayes,
    manrique1998simulation, andrieu2002particle}
  \end{itemize}
 \begin{figure}[ht]
   \centering
   \includegraphics[width=0.45\textwidth]{figure/glasby-nevison}
   \caption{The essential idea of Dynamic Tobit models is well illustrated by
     this figure from \citep{glasbey1997rainfall}, which studied precipitation
     data. \label{fig:glasby-nevison}}
 \end{figure}
\end{frame}

\begin{frame}
  \frametitle{Conclusion}
 \begin{itemize}
 \item We surveyed a variety of methods for understanding changes in time series
   dynamics
\item Outside of \citep{digiulio2015temporal}, we have rarely provided answers to this
  set of questions, though there are many tools available
\item It would be valuable to consolidate, or provide guidelines for choosing
  between, these different approaches

 \end{itemize} 
\end{frame} 
\bibliographystyle{plainnat}
\bibliography{refs.bib}
 
\end{document}
