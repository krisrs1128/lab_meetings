\documentclass{beamer}
\usetheme{Warsaw}
\usepackage{natbib}
\usepackage{graphicx}
% ------------------------------------------------------------------------
% Packages
% ------------------------------------------------------------------------
\usepackage{amsmath}

% ------------------------------------------------------------------------
% Macros
% ------------------------------------------------------------------------
%~~~~~~~~~~~~~~~
% List shorthand
%~~~~~~~~~~~~~~~
\newcommand{\BIT}{\begin{itemize}}
\newcommand{\EIT}{\end{itemize}}
\newcommand{\BNUM}{\begin{enumerate}}
\newcommand{\ENUM}{\end{enumerate}}
%~~~~~~~~~~~~~~~
% Text with quads around it
%~~~~~~~~~~~~~~~
\newcommand{\qtext}[1]{\quad\text{#1}\quad}
%~~~~~~~~~~~~~~~
% Shorthand for math formatting
%~~~~~~~~~~~~~~~
\newcommand\mbb[1]{\mathbb{#1}}
\newcommand\mbf[1]{\mathbf{#1}}
\def\mc#1{\mathcal{#1}}
\def\mrm#1{\mathrm{#1}}
%~~~~~~~~~~~~~~~
% Common sets
%~~~~~~~~~~~~~~~
\def\reals{\mathbb{R}} % Real number symbol
\def\integers{\mathbb{Z}} % Integer symbol
\def\rationals{\mathbb{Q}} % Rational numbers
\def\naturals{\mathbb{N}} % Natural numbers
\def\complex{\mathbb{C}} % Complex numbers
\def\simplex{\mathcal{S}} % Simplex
%~~~~~~~~~~~~~~~
% Common functions
%~~~~~~~~~~~~~~~
\renewcommand{\exp}[1]{\operatorname{exp}\left(#1\right)} % Exponential
\def\indic#1{\mbb{I}\left({#1}\right)} % Indicator function
\providecommand{\argmax}{\mathop\mathrm{arg max}} % Defining math symbols
\providecommand{\argmin}{\mathop\mathrm{arg min}}
\providecommand{\arccos}{\mathop\mathrm{arccos}}
\providecommand{\asinh}{\mathop\mathrm{asinh}}
\providecommand{\dom}{\mathop\mathrm{dom}} % Domain
\providecommand{\range}{\mathop\mathrm{range}} % Range
\providecommand{\diag}{\mathop\mathrm{diag}}
\providecommand{\tr}{\mathop\mathrm{tr}}
\providecommand{\abs}{\mathop\mathrm{abs}}
\providecommand{\card}{\mathop\mathrm{card}}
\providecommand{\sign}{\mathop\mathrm{sign}}
\def\rank#1{\mathrm{rank}({#1})}
\def\supp#1{\mathrm{supp}({#1})}
%~~~~~~~~~~~~~~~
% Common probability symbols
%~~~~~~~~~~~~~~~
\def\E{\mathbb{E}} % Expectation symbol
\def\Earg#1{\E\left[{#1}\right]}
\def\Esubarg#1#2{\E_{#1}\left[{#2}\right]}
\def\P{\mathbb{P}} % Probability symbol
\def\Parg#1{\P\left({#1}\right)}
\def\Psubarg#1#2{\P_{#1}\left[{#2}\right]}
\def\Cov{\mrm{Cov}} % Covariance symbol
\def\Covarg#1{\Cov\left[{#1}\right]}
\def\Covsubarg#1#2{\Cov_{#1}\left[{#2}\right]}
\def\Var{\mrm{Var}}
\def\Vararg#1{\Var\left(#1\right)}
\def\Varsubarg#1#2{\Var_{#1}\left(#2\right)}
\newcommand{\family}{\mathcal{P}} % probability family
\newcommand{\eps}{\epsilon}
\def\absarg#1{\left|#1\right|}
\def\msarg#1{\left(#1\right)^{2}}
\def\logarg#1{\log\left(#1\right)}
%~~~~~~~~~~~~~~~
% Distributions
%~~~~~~~~~~~~~~~
\def\Gsn{\mathcal{N}}
\def\Ber{\textnormal{Ber}}
\def\Bin{\textnormal{Bin}}
\def\Unif{\textnormal{Unif}}
\def\Mult{\textnormal{Mult}}
\def\Cat{\textnormal{Cat}}
\def\Gam{\textnormal{Gam}}
\def\InvGam{\textnormal{InvGam}}
\def\NegMult{\textnormal{NegMult}}
\def\Dir{\textnormal{Dir}}
\def\Lap{\textnormal{Laplace}}
\def\Bet{\textnormal{Beta}}
\def\Poi{\textnormal{Poi}}
\def\HypGeo{\textnormal{HypGeo}}
\def\GEM{\textnormal{GEM}}
\def\BP{\textnormal{BP}}
\def\DP{\textnormal{DP}}
\def\BeP{\textnormal{BeP}}
%~~~~~~~~~~~~~~~
% Theorem-like environments
%~~~~~~~~~~~~~~~

%-----------------------
% Probability sets
%-----------------------
\newcommand{\X}{\mathcal{X}}
\newcommand{\Y}{\mathcal{Y}}
\newcommand{\D}{\mathcal{D}}
\newcommand{\Scal}{\mathcal{S}}
%-----------------------
% vector notation
%-----------------------
\newcommand{\bx}{\mathbf{x}}
\newcommand{\by}{\mathbf{y}}
\newcommand{\bt}{\mathbf{t}}
\newcommand{\xbar}{\overline{x}}
\newcommand{\Xbar}{\overline{X}}
\newcommand{\tolaw}{\xrightarrow{\mathcal{L}}}
\newcommand{\toprob}{\xrightarrow{\mathbb{P}}}
\newcommand{\laweq}{\overset{\mathcal{L}}{=}}
\newcommand{\F}{\mathcal{F}}
\def\colarg#1#2{\textcolor[HTML]{#1}{#2}}

\setbeamersize{text margin left=5pt,text margin right=5pt}

\setbeamerfont{institute}{size=\fontsize{7pt}{8pt}}
\setbeamerfont{date}{size=\fontsize{0pt}{0pt}}

\title{Text Modeling meets the Microbiome}
\author{Kris Sankaran and Susan P. Holmes}
\institute{Department of Statistics, Stanford University}
\date{}
\begin{document}

\begin{frame}
  \maketitle
\begin{figure}
  \centering
  \includegraphics[width=0.5\textwidth]{figure/title.jpg}
\end{figure}

\end{frame}

\begin{frame}
  \frametitle{Microbiome vs. Text Analysis}
  \begingroup
  \fontsize{10pt}{10pt}\selectfont
  {\ttfamily
  \def\arraystretch{1.2}
  \setlength{\tabcolsep}{0.2em} % for the horizontal padding
  \begin{table}
    % latex table generated in R 3.3.2 by xtable 1.8-2 package
% Fri Sep  8 13:46:28 2017
\begin{tabular}{rlrrrrrrrr}
  \hline
index & book & elizabeth & darcy & bennet & miss & jane & bingley & time \\ 
  \hline
  0 & P \& P &   0 &   0 &   4 &   0 &   1 &   3 &   0 \\ 
    1 & P \& P &   1 &   0 &   5 &   0 &   1 &   4 &   0 \\ 
    2 & P \& P &   0 &   0 &   6 &   0 &   0 &   5 &   1 \\ 
    3 & P \& P &   1 &   4 &   5 &   1 &   0 &   9 &   1 \\ 
    4 & P \& P &   3 &   3 &   5 &   4 &   4 &   5 &   3 \\ 
    5 & P \& P &   3 &   0 &   0 &   2 &   1 &   6 &   1 \\ 
    6 & P \& P &   0 &   6 &   6 &   7 &   1 &   5 &   1 \\ 
   \hline
\end{tabular}
 
    \\[12pt]
    % latex table generated in R 3.3.2 by xtable 1.8-2 package
% Fri Sep  8 14:31:53 2017
\begin{tabular}{rlrrrrr}
  \hline
time & subject & Unc06grq & Unc09fy6 & Unc06bhm & Unc06g1h & Unc06af7 \\ 
  \hline
  0 & D & 791 &   0 &  79 & 108 &  11 \\ 
    1 & D & 1616 &   0 & 1413 & 192 &  31 \\ 
    2 & D & 1323 &   0 & 915 & 165 &  23 \\ 
    3 & D & 1846 &   0 & 1366 & 170 &  31 \\ 
    4 & D & 2314 &   0 & 689 & 135 &  26 \\ 
    5 & D & 2244 &   0 & 776 & 310 & 175 \\ 
    6 & D & 1652 &   0 & 609 & 235 & 181 \\ 
   \hline
\end{tabular}
 
  \end{table}
  }
  \endgroup
\end{frame}

\begin{frame}
  \frametitle{Microbiome vs. Text Analysis}
 \begin{itemize}
 \item Each field could benefit by recognizing tools developed in the other
 \item Text analysis methods have been used successfully in the microbiome
   literature before \citep{cai2017learning, shafiei2015biomico,
     chen2012estimating, yan2017metatopics}
 \item Similarities
   \begin{itemize}
   \item Form: High-dimensional, sparse, count matrices, with supplemental
     information (``metadata'')
   \item Goals: Information-dense data compression
   \end{itemize}
 \item Differences
   \begin{itemize}
   \item Form: Sentences vs. networks
   \item Goals: Predictive black-boxes vs. scientific inference
   \end{itemize}
 \end{itemize} 
\end{frame}

\begin{frame}
  \frametitle{Microbiome vs. Text Analysis}
\begin{figure}[ht]
  \centering
  \includegraphics[width=1\textwidth]{figure/text_vs_microbiome}
  \caption{A translation, to help develop the analogy. \label{fig:label} }
\end{figure}

\end{frame}

\begin{frame}
  \frametitle{Example: Latent Dirichlet Allocation (LDA)}
\begin{itemize}
  \item Introduced in \citep{blei2003latent}
  \item Mixed-membership: Middle-ground between clustering (well-separated
    types) and ordination (continuous gradient)
  \item Predictive checks: Probabilistic formalism allows incorporation of
    study-specific structure and model assessment
\end{itemize}  
\begin{figure}[ht]
  \centering
  \includegraphics[width=0.37\textwidth]{figure/simplex}
  \caption{A toy representation of the mixed-membership idea.\label{fig:label} }
\end{figure}

\end{frame}

\begin{frame}
  \frametitle{Example: LDA}
 \begin{figure}[ht]
   \centering
   \includegraphics[width=0.9\textwidth]{figure/visualize_lda_theta_boxplot-F}\\
   \caption{The positions of points in the data of
     \citep{dethlefsen2011incomplete}. \label{fig:label} }
 \end{figure}
\end{frame}

\begin{frame}
  \frametitle{Example: LDA}
 \begin{figure}[ht]
   \centering
   \includegraphics[width=0.86\textwidth]{figure/visualize_lda_beta-F}
   \caption{An interpretation of the corners for the data of \citep{dethlefsen2011incomplete}. \label{fig:label} }
 \end{figure}
\end{frame}

\begin{frame}
  \frametitle{Model Assessment}
\begin{itemize}
\item Simulation and assessment allows iterative improvement
\item Poor fit can be used to identify interesting outliers
\end{itemize}
  \begin{figure}[!p]
    \centering
    \includegraphics[width=0.6\textwidth]{figure/posterior_check_ts-F}
    \caption{Posterior simulated trajectories for a subset of species, according
      to two models.}
  \end{figure} 
\end{frame}

\begin{frame}
  \frametitle{Conclusion}
 \begin{itemize}
 \item Distillation $>$ Addition: We have studied applications of common text
   analysis algorithms to the study microbiome-specific questions
 \item Reproducibility: All of our data and code is public
   \url{https://github.com/krisrs1128/microbiome_plvm/}
 \item Preprint: ``Latent Variable Modeling for the Microbiome''
   \url{https://arxiv.org/abs/1706.04969}
 \end{itemize} 
\end{frame}

\begin{frame}
\bibliographystyle{plainnat}
\bibliography{refs.bib}
\end{frame}

\end{document}
